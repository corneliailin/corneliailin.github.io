% File produced by Jeremy West

\documentclass[margin,line]{res}                          % Custom class: "res.cls" file needed (included)

% Document:
\usepackage{setspace}                                     % Allows for custom margins, etc.
\usepackage{fullpage}                                       % Use the full page
\usepackage{outlines}				       % Paragraphs
\usepackage[dvipsnames]{xcolor}		       % Add colors
\usepackage[english]{babel}                             % English letters (change here for a different language)


% Customize document size:
\oddsidemargin -.5in
\evensidemargin -.5in
\textwidth=6.0in
\itemsep=0in
\parsep=0in
\setlength{\pdfpagewidth}{\paperwidth}
\setlength{\pdfpageheight}{\paperheight}
\addtolength{\topmargin}{-.3in}
\addtolength{\textheight}{0.6in} 

% Font/text:
\usepackage[latin9]{inputenc}                             % Font definition and input type
\usepackage{bm}
\usepackage[T1]{fontenc}                                  % Font output type 
\usepackage{lmodern}                                      % Latin Modern fonts
\usepackage{textcomp}                                     % Supports many additional symbols
\usepackage{color}                                        % Enables colored text
\definecolor{darkblue}{rgb}{0.0,0.0,0.66}                 % Custom color: dark blue
\usepackage[hyperfootnotes=false,bookmarksopen]{hyperref} % Enable hyperlinks, expand menu subtree
\hypersetup{                                              % Custom hyperlink settings
    pdffitwindow=false,                                   % Window fit to page when opened
    pdfstartview={XYZ null null 1.00},                    % Fits the zoom of the page to 100%
    pdfnewwindow=true,                                    % Links in new window
    colorlinks=true,                                      % false: boxed links; true: colored links
    linkcolor=darkblue,                                   % Color of internal links
    citecolor=darkblue,                                   % Color of links to bibliography
    urlcolor=darkblue,                                    % Color of external links
    pdfauthor = {Your name},                              % PDF metadata - set within hypersetup
    pdfkeywords = {Institution, economics, etc.},         % PDF metadata - set within hypersetup
    pdftitle = {Your name: Curriculum Vitae},             % PDF metadata - set within hypersetup
    pdfsubject = {Curriculum Vitae},                      % PDF metadata - set within hypersetup
    pdfpagemode = UseNone}                                % PDF metadata - set within hypersetup

% Miscellaneous:
\usepackage{datetime}                                     % Custom date format for date field
\newdateformat{mydate}{\monthname[\THEMONTH] \THEYEAR}    % Defining month year date format

% Customize page headers:
\usepackage{fancyhdr}                                     % Used for custom page headers
\pagestyle{fancy}
\fancyhf{}
\renewcommand{\headrulewidth}{0.5pt} 
\rhead{\footnotesize Cornelia Ilin: Curriculum Vitae (\mydate \today), Page \thepage} %header at the right
\headsep = 0.5cm
% FIRST PAGE ONLY (redefine the plain pagestyle
\fancypagestyle{plain}{
\fancyhf{}
\renewcommand{\headrulewidth}{0pt} 
\headsep = 0.0cm
\rhead{\footnotesize Curriculum Vitae updated \mydate \today}
}

% Define list environments:
\newenvironment{list1}{
  \begin{list}{\ding{113}}{%
      \setlength{\itemsep}{0in}
      \setlength{\parsep}{0in} \setlength{\parskip}{0in}
      \setlength{\topsep}{0in} \setlength{\partopsep}{0in} 
      \setlength{\leftmargin}{0.17in}}}{\end{list}}
\newenvironment{list2}{
  \begin{list}{$\bullet$}{%
      \setlength{\itemsep}{0in}
      \setlength{\parsep}{0in} \setlength{\parskip}{0in}
      \setlength{\topsep}{0in} \setlength{\partopsep}{0in} 
      \setlength{\leftmargin}{0.2in}}}{\end{list}}

%%%%%%%%%%%%%%%%%%%%%%%%%%%%%%%%%%%%%%%%%%%%

\begin{document}
    
\name{ {\LARGE Cornelia Ilin} \vspace*{.1in}}

\begin{resume}
\thispagestyle{plain} % to use first page footer

\section{\sc Contact Information}
\vspace{.05in}
\begin{tabular}{@{}p{0.20in}p{3.75in}p{2.75in}}
 & 3416 Plateau Dr & (608) 338-4844\\   
 & Belmont, CA  94002& \href{mailto: Your email}{cornelia.ilin@berkeley.edu}\\            
 & & \href{http://corneliailin.github.io}{http://corneliailin.github.io}\\         
 &  & \\         
\end{tabular}

\section{\sc }
\begin{tabular}{@{}p{0.20in}p{3.75in}p{2.75in}}
 & \textbf{Citizenship: U.S. permanent resident}
\end{tabular}
\vspace{0.1cm}


\section{\sc Profile}
\begin{list2}
\item[] -----------------------------------------------------------------------------------------------------------------------------
\item[] I am an Assistant Professor of Practice at UC Berkeley, School of Information, where I teach Applied Machine Learning (ML) and Capstone classes to MS-level students. My research sits at the intersection of health and the environment. In my research, I combine \textbf{ML} with \textbf{geospatial} and \textbf{causal inference} methods. I draw from more than 13 years of experience across academia, industry, and consulting, with highlights including:
\end{list2}

\vspace{0.2 cm}
\begin{outline}

\1 Highly rated Data Science Professor. Presentation and writing skills (I attend conferences and write papers for journal publications). Collaboration skills (direct interaction with faculty, postdocs, and graduate students).

\1 Computer languages include \textbf{Python}, GIS, SQL, Bash shell; version control with Git(Hub)

\1 Experience with state-of-the art Natural Language Processing (NLP) and Large Language Models (LLM) architectures

\1 Experience with big data analytics: analyzed data using Hadoop/HDFS and Dask frameworks on CloudLab.us clusters and Google Cloud Platform (GCP)

\1 Experience with end-to-end ML pipeline development on GCP

\1 Experience with packages such as TensorFlow2, SHAP for explainable AI, Scikit-Learn, Statsmodels, Geopandas, OSMnx, Rasterio

\1 Experience with IRB applications and working with \textbf{HIPAA protected health data}, including claims and electronic health records (e.g., diagnosis, lab values, clinician notes, images).

\1 Two years of \textbf{industry and consulting experience} with contributions to high-profile litigation cases in the healthcare industry

\end{outline}

\begin{list2}
\item[] -----------------------------------------------------------------------------------------------------------------------------
\end{list2}


\section{\sc Research Affiliations}
\begin{list2}
\item[] \textcolor{Black}{\textbf{UC-Berkeley}, Global Policy Lab} \hspace{7.2cm}2023 - present

\begin{itemize}
\item The Aerial History Project: the objective is to convert 1.6 million historical aerial photos taken between the 1940s and 1990s into a data set comparable to modern satellite imagery. The final product will provide input to various downstream tasks, such as understanding how climatic shifts have mobilized populations.
\item Data: The aerial photos were collected by the British Directorate of Overseas Surveys while mapping what was formerly known as the British Empire. It covers large parts of Africa, the Far East, and the islands of the Caribbean Sea, Atlantic, and Pacific Oceans.
\item Methods: computer vision algorithms for stitching and georeferencing of mosaics, and predictions
\end{itemize}
\item[]


\item[] \textcolor{Black}{\textbf{UW-Madison}, Environmental Research Group} \hspace{4.9cm}2020 - present
\begin{itemize}
\item Project 1: Understand the effects of wildfire exposure in-utero and first years of life on the likelihood of respiratory, cancer, and diabetes conditions in one's emergency room or inpatient visit later in life. A version of this project won the School of Information's \textbf{MIDS Hal Varian} award in the Fall of 2022.
\item Project 2: Develop Ped-BERT, a state-of-the-art deep learning model that accurately predicts the likelihood of 100+ conditions in a pediatric patient's next medical visit.
\item Data: emergency room and inpatient visits, vital statistics, wind and wildfire data
\item Methods: record linkage, causal analysis with instrumental variables, bidirectional encoder representations from transformers (BERT)
\end{itemize}
\item[]
\end{list2}





\section{\sc Professional Experience}
\begin{list1}

\item[] \textcolor{Black}{\textbf{UC-Berkeley}, School of Information, \textbf{Assistant Professor of Practice}} \hspace{1.0cm}2020 - present
\item[]

\item[] \textcolor{Black}{\textbf{UC-Berkeley}, School of Information, \textbf{Lecturer}} \hspace{5.5cm}2020 - 2023
\begin{itemize}
\item DataSci 207: Applied ML (course coordinator). Topics include linear regression, logistic regression, decision trees, random forests,  unsupervised learning (clustering and dimensionality reduction), deep neural networks (FNN, CNN, RNN/LSTM, Transformers), embeddings, transfer learning, model interpretability and fairness
\item DataSci 210: Capstone. A project-based course fusing core data science and soft skills learned throughout the MIDS program.
\end{itemize}
\item[]


\item[] \textcolor{Black}{\textbf{Stanford University}, RegLab, \textbf{Research Scientist}} \hspace{4.7cm}2021 - 2022

\begin{itemize}
\item Project (in partnership with the EPA): explore how machine learning can be used to protect human health with a focus on environmental justice and health outcomes in California. 
\item Data: Satellite imagery to detect Concentrated Animal Feeding Operations (CAFOs), atmospheric data (wind patterns from NASA's MERRA-2 product), water pollutant discharge monitoring data (ICIS-NPDES), hospital/ER data (CDPH), census tract data (Geolytics Neighborhood Change Database)
\item Methods: computer vision algorithms for time series satellite images classification to detect building construction and expansion, causal inference with instrumental variables
\end{itemize}
\item[]


\item[] \textcolor{Black}{\textbf{UC-Berkeley}, School of Information, \textbf{Postdoctoral Fellow}} \hspace{3.7cm}2020 - 2021

\begin{itemize}
\item Advisor: Joshua Blumenstock, Ph.D.
\item Project 1: Provide real-time feedback for managing the spread of COVID-19 at local, national, and global scale. Focus on the impact of non-pharmaceutical policies on human mobility, and the usefulness of cellphone data in predicting the spread of the pandemic.

\item Project 2 (in partnership with the CDC): Empirically estimate the impact of changes in non-pharmaceutical policy interventions, mobility, and other avoidance behaviors on growth rate of COVID-19 cases, COVID-19 deaths, and economic output for all countries in the world where GDP data is available.

\item Data: COVID-19 cases and deaths (John Hopkins CSSE), human mobility data (Google, Facebook, SafeGraph, InfoGroup),  non-pharmaceutical policy interventions (CDC and other sources), quarterly GDP data.

\item Methods: Causal inference, predictive analysis.


\end{itemize}
\item[]


\item[] \textcolor{Black}{\textbf{UW-Madison}, Department of Applied Economics, \textbf{Faculty Associate}}  \hspace{2.0cm}2018 - 2020
\begin{itemize}
\item Teaching: Object Oriented Programming and Data Analytics with Python; Practicum for Applied Economists
\item Topics: Data types, functions, classes, exceptions, IO files, data visualization, descriptive statistics, causal inference, cloud computing (incl. Bash), GIS with Python.
\item Assistant Program Director: M.S. in Quantitive and Applied Economics program.
\end{itemize}
\item[]


\item[] \textcolor{Black}{\textbf{Analysis Group, Inc.}, Menlo Park, CA, \textbf{Associate Economist}} \hspace{2.8 cm} 2017 - 2018
\begin{itemize}
\item Litigation consulting: Contributed to several high-profile litigation cases in the healthcare industry (e.g., Des Roches, et al. v. Blue Shield and Magellan).
\item Research: Contributed to manuscripts and posters documenting the effectiveness of leptin replacement therapy in treating lipodystrophy.
\item Data: Claims (e.g., mental health, substance abuse), quasi-experimental and surveys.
\item Methods: Surveys, discrete choice analysis (multinomial logistic and hierarchical Bayesian regressions), matching algorithms, Cox hazard models.
\end{itemize}
\item[]

\item[] \textcolor{Black}{\textbf{UW-Madison}, Department of Applied Economics, \textbf{Research Assistant}} \hspace{1.703cm} 2012 - 2017
\item[]


\item[] \textcolor{Black}{\textbf{University of Z\"urich},  Department of Economics, \textbf{Research Assistant}} \hspace{3.0cm} 2011
%\begin{itemize}
%\item \textbf{Advisor}: Armin Schmutzler, Ph.D.
%\end{itemize}
\item[]

\item[] \textcolor{Black}{\textbf{EPFL}, Department of Computer Science, \textbf{Research Assistant}} \hspace{4.35cm} 2010
%\begin{itemize}
%\item \textbf{Advisor}: Panos Papadimitratos, Ph.D.
%\end{itemize}
\item[]


\item[] \textcolor{Black}{\textbf{DHL} European Headquarters, Belgium, \textbf{Intern}} \hspace{6.75 cm} 2009
%\begin{itemize}
%\item \textbf{Role}:  Provided weekly reporting on key metrics and analyses to managers in the Procurement Department.
%\end{itemize}
\item[]

\end{list1}

\section{\sc Education}
\begin{list1}
\item[]  UW-Madison, Ph.D. in  Applied Economics  \hspace{6.3 cm} 2012 - 2017
\item[]  University of Lausanne, Switzerland, M.S. in Economics \hspace{4.5cm} 2009 - 2011
\item[]  Academy of Economic Studies of Bucharest, Romania, B.S. in Economics \hspace{1.85cm} 2004 - 2008
\end{list1}


%\section{\sc Research and Teaching Fields}
%\begin{list2}
%\item[] I use statistical, machine learning, and record linkage methods to study the economic consequences of industrial organization and health-related problems. I teach Applied Machine Learning for the MS in Data Science program at UC Berkeley.
%\end{list2}

\section{\sc Journal Publications}
\begin{list2}
\item[]Improving Nonalcoholic Fatty Liver Disease Classification Performance With Latent Diffusion Models (\textit{under review, Nature - Scientific Reports}, with R. Hardy, R. Mitchell, J. Klepich, S. Hall, J. Villareal, 2023)
\\
\item[] Global Health and Economic Impacts of Behavior Change During the COVID-19 Pandemic (\textit{under review, Nature}, with J. Tseng, K.C. Coy, A.C. Ewing, T. Chong, S.M. Marks, I. Bolliger, N.M. Gonzalez, K. Bell, A.J. Hakim, S. Hsiang,  2021)
\\
\item[] Public Mobility Data Enables COVID-19 Forecasting and Management at Local and Global Scales (\textit{Nature - Scientific Reports, volume 11, article number: 13531}, with S. Annan-Phan, X.H. Tai,  S. Mehra, S. Hsiang, J. Blumenstock, 2021).
\\
\item[] Competition, Price Dispersion and Capacity Constraints: The Case of the U.S. Corn Seed Industry, \textit{European Review of Agricultural Economics, 2021} (with G. Shi).

\end{list2}

$\; \; \; \;$ \textbf{in preparation}:
\begin{list2}
\item[] Ped-BERT: Early Detection of Disease for Pediatric Care (2022)
\\

\item[] The Role of Birth and Contemporaneous Pollution Exposure on Health Outcomes. Evidence from California (with  D. Phaneuf, 2020). 
\\
\end{list2}


\section{\sc Manuscripts and Posters}
\begin{list2}
\item[] Longitudinal Matching. A Method for Generating Comparable Samples of Treatment and Treatment-Naive Patients with Progressive Conditions (Analysis Group, 2018).
\\
\item[] Effect of Leptin Replacement Therapy on Survival and Disease Progression in Generalized and Partial Lipodystrophy (Analysis Group, 2018). 
\\
\item[] Patient Quality of Life and Benefits of Leptin Replacement Therapy in Generalized and Partial Lipodystrophy (Analysis Group, 2018).
\\
\end{list2}


\section{\sc Other Teaching Experience}
\begin{list2}
\item[] UW-Madison:
\item[] TA, World Hunger and Malnutrition: Spring 2017
\item[] TA,  Applied Econometric Analysis I: Fall 2016
\item[] TA,  Applied Microeconomic Theory:  Fall 2014
\item[] Lecturer, Math Camp for Incoming M.S. and Ph.D. Students: Summer 2014 
\end{list2}

\section{\sc Fellowships, Scholarships and Grants}
\begin{list2}
\item[] PDF Grant, University of California - Berkeley, 2023
\item[] Science of ADRD Workshop (competitive), University of Southern California, 2022
\item[] Research Grant, American Bar Association, Section of Antitrust Law, 2016
\item[] Ph.D. Summer Program (competitive), Edgeworth Economics, Washington, DC, 2016
\item[] Kenneth and Pauline Parsons Graduate Fellowship Fund, UW-Madison, 2016
\item[] Best Paper Presentation Award, UW-Madison, 2016
\item[] SASC Graduate Funds, University of Lausanne, 2010 - 2011
\item[] Hessen Summer School (competitive), Goethe University of Frankfurt am Main, Germany, 2008
\item[] WU Summer School (competitive), Vienna University of Economics and Business, Austria, 2007
\item[] Excellency in Research Award, Academy of Economic Studies of Bucharest, Romania, 2007

\end{list2}

\section{\sc Seminar and Conference Presentations}
\begin{list2}
%\item[] Stanford Global Health Research Convening, 2022
\item[] Stanford Maternal and Child Health Research Institute Symposium, 2021, 2022
\item[] Association of Environmental and Resource Economics, 2020
\item[] UW-Madison, Healthcare Group seminar, 2016, 2019
\item[] University of Connecticut, 2017
\item[] European Association for Research in IO (Rising Stars section), Lisbon, Portugal, 2016
\item[] AAEA Meetings, Boston, Massachusetts, 2016
\end{list2}

\section{\sc Professional Activities}
\begin{list2}
\item[] Reviewer for Nature Scientific Reports, 2022 - present
\item[] Reviewer for the American Public Health Association (APHA), 2019 - present
\item[] Social Chair, THC Club of AAE Department, UW-Madison, 2015 - 2016
\item[] Seminar Organizer, THC Club of AAE Department, UW-Madison, 2014 - 2015
\end{list2}


\section{\sc Language Skills}
\begin{list2}
\item[] Romanian (native), English (fluent), French (basic)
\end{list2}

%\section{\sc References} 
%\vspace{.05in}
%\begin{tabular}{@{}p{0.20in}p{2.5in}p{3in}}
%& Joshua Blumenstock, Ph.D. (Advisor)                                                                            & Guanming Shi, Ph.D. (Advisor) \\
%& Associate Professor												    & Professor \\
%& School of Information                                                           					    & Department of Applied Economics  \\
%& UC-Berkeley                  						       						    & UW-Madison \\
%& (510) 642-4583                          								                     & (608) 263-6250 \\
%& \href{mailto:email@email.edu}{jblumenstock@berkeley.edu}                   		    & \href{mailto:email@email.edu}{gshi@wisc.edu} \\
%&                         											                     & \\
%& Daniel Phaneuf, Ph.D.                                                                                                    & \\
%& Henry C. Taylor Professor and Chair                                                                              & \\
%&  Department of Applied Economics                                                                                & \\
%& UW-Madison                    						         				    & \\
%& (608) 262-4908                           								                     & \\
%& \href{mailto:email@email.edu}{dphaneuf@wisc.edu} & \href{mailto:email@email.edu}{} \\
%&                                      & \\
%\end{tabular}



\end{resume}
\end{document}